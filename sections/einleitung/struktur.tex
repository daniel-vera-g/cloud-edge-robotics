\subsection{Struktur der Arbeit} % (fold)
\label{sub:Struktur der Arbeit}

Diese Arbeit ist in drei Teilen aufgebaut. Im ersten Teil wird eine Hinführung auf die für diese Arbeit nötigen Grundlagen gegeben. Dazu werden Konzepte aus den Bereichen der Cloud und dem Edge eingeführt. Es werden ebenfalls die Eigenschaften von Robotik Systemen beschrieben und mit dem \acrlong{ros} das aktuell gängige System zur Nutzung in der Robotik vorgestellt. Um die Schnittstelle zum Cloud und Edge Bereich herzustellen, wird hier ein Fokus auf die verwendeten Kommunikationsprotokolle in der Robotik gesetzt.\\
Der Hauptteil Gliedert sich in vier Abschnitte. Zu Beginn wird erklärt, wie die Zusammenführung von der Cloud, dem Edge und Robotern funktionieren kann und welche Vorteile man aus dem Zusammenschluss bekommen kann. Im folgenden, wird ein Blick auf die Einbindung der Roboter in das Gesamtsystem durch das Zenoh Protokoll gesetzt. Nach den eher spezielleren Themen, wird in den letzten beiden Abschnitten vom Hauptteil eine übergeordnete Sicht auf das Thema Cloud und Edge Robotics gemacht. Dazu werden allgemeine Architekturen betrachtet. Darauf aufbauend, werden verschiedene Use Cases vorgestelt die eine praktische Anwendung zeigen sollen.\\
Die Arbeit wird mit dem Fazit abgeschlossen. Hier wird ein Ausblick auf Technologien und die Weiterentwicklung in der Forschung und Entwicklung von Bereichen im Cloud und Edge Robotics Spektrum gemacht. Schließlich wird das ganze durch ein Fazit über die behandelten Themen beendet.

% subsection Struktur der Arbeit (end)

