\subsection{Motivation}

Wie man der Einleitung entnehmen konnte, steigt die Nutzung von Robotern kontinuierlich. Vor allem in der Industrie stellt sich dabei die Frage, wie man eine große Anzahl an Robotern in Produktion skalieren, in instabilen Umgebungen produktiv betreiben oder mit etablierten Cloud Lösungen integrieren kann. Beim Versuch diese Fragen zu beantworten, stößt man zwangsläufig auf Herausforderungen die im Zusammenhang mit der Nutzung von Robotern untereinander oder in Verbindung mit externen Services stehen.\\
Bei den genannten Herausforderungen geht es um verschiedene Bereiche. Zum einen natürlich um die zusammenarbeit Heterogener Systeme untereinander. Diese müssen fähig sein, miteinander interagieren zu können und zu kommunizieren. Wenn man noch die Cloud und Edge Server miteinbezieht, kommen noch weitere Systeme dazu die man berücksichtigen muss. Zum anderen, kommen hier auch die Cloud und Edge Systeme durch ihre unterschiedlichen Eigenschaften zur Geltung. Im Cloud Bereich, hat man dabei homogene Ressourcen die kurze Zyklen durchlaufen. Außerdem hat man eine hohe Verfügbarkeit. Im Edge Bereich ist genau das Gegenteil der Fall. Ressourcen sind sehr Heterogen, haben lange Lebenszyklen und die Verfügbarkeit ist sehr Variabel.\\
Neben der Art der Ressourcen, sind auch die Eigenschaften des Netzwerkes eine konstante Herausforderung. Ziel ist es dabei einen hohen Durchsatz sowie eine niedrige Latenz zu haben. Dies ist aber aufgrund der Art des Netzwerkes oft nicht möglich. Vor allem im Bereich der Robotik, muss man mit Drahtlosen, unzuverlässige Verbindungen umgehen.\\
Zu den Netzwerkherausforderungen kommen schließlich noch die Herausforderungen die sich durch die Topologie ergeben. Diese ergeben sich durch die große Anzahl an Beweglichen Knoten die im System vorhanden sind und die Geographische Verteilung die diese haben können.\\
Die Fragestellung dieser Arbeit setzt sich also aus den hier geschilderten Fragen zusammen. Dabei ist zu klären wie man mit den verschiedenen Herausforderungen umgehen kann und mit welcher Lösung man die verschiedenen Komponenten am besten in Einklang bringt.\\
