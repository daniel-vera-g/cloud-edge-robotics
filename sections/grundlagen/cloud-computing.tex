\subsection{Cloud Computing}

Cloud Computing ist ein sehr großer und weitreichender Begriff der in den letzten Jahren sehr stark gewachsen ist. Rimal et al. \cite{rimalArchitecturalRequirementsCloud2011} beschreibt Cloud Computing folgendermaßen: "Cloud Computing is a model of service delivery and access where dynamically scalable and virtualized resources are provided as a service over the Internet.". Wie in der Definition erwähnt, bietet Cloud computing einen großen Vorteil gegenüber traditionellen, selbst betriebene Servern. Bei Cloud Computing hat man dabei zum einen die Möglichkeit Ressourcen nahezu beliebig zu nutzen und nach Bedarf Skalieren. Zum anderen bietet Cloud Computing nicht nur Zugang zu reiner Rechenleistung oder Speicherplatz, sondern auch zu spezialisierten Services wie Machine Learning spezifische Anwendungen oder Big Data Lösungen\cite{antonopoulosCloudComputingPrinciples2017}.\\

Cloud Computing Services lassen sich in drei verschiedene Modele einteilen \cite{antonopoulosCloudComputingPrinciples2017}:

\begin{enumerate}
  \item Infrastructure as a Service (IaaS): IaaS ist die grundlegendste Art Ressourcen aus der Cloud zu beziehen. Dabei bezieht man Ressourcen wie Rechen oder Speicherleistung direkt über das Internet und kann darüber verfügen. Ein Vorteil von Infrastructure as a Service ist, dass es möglich ist nur für die genutzten Ressourcen zu zahlen.
  \item Platform as a Service (PaaS): PaaS befindet sich eine Stufe über dem IaaS und bietet Nutzern eine Platform an die all jene Systeme schon enthält die man für den jeweiligen Use Case braucht. Beim Platform as a Service, muss man sich keine Gedanken über die Planung von Ressourcen oder das aufsetzen von Systemen machen. Dies wird alles von dem jeweiligen Anbieter übernommen und spart dem Endanwender sowohl Zeit als auch know how über die darunterliegenden Systemen.
  \item Software as a Service (SaaS): SaaS ist die höchste Stufe des Cloud Computing und bietet dem Nutzer fertige Softwarelösungen an. Eine Besonderheit hier ist, dass ich viele verschiedene Nutzer die Ressourcen einer Anwendung teilen die sie gemeinsam nutzen. Anwender haben hier den leichtesten Zugang zu einer bestimmten Lösung, jedoch die kleinste individuelle Anpassbarkeit.
\end{enumerate}

Neben den Arten der Cloud Services, gibt es ebenfalls verschiedene Modelle und Cloud Lösungen bereitzustellen \cite{antonopoulosCloudComputingPrinciples2017}:

\begin{enumerate}
  \item Public Cloud: Dies ist die populärste Art Cloud Lösungen einzusetzen und richtet sich an die Allgemeinheit der Nutzer. Cloud Computing Ressourcen sind dabei für jeden über öffentliche Anbieter verfügbar und können je nach Bedarf skaliert werden.
  \item Private Cloud: Das Model der Privaten Cloud wird meistens von Unternehmen und Organisationen genutzt. Dabei werden Ressourcen und Netzwerkzugänge von der jeweiligen Institution selbst verwaltet. Private Cloud Modelle haben Vorteile, wenn die Sicherheit und Gesetzliche Vorgaben hohe Priorität haben. Das Hosting, wird bei Private Cloud Modelle oft durch die die Nutzer selber betrieben.
  \item Hybrid Cloud: Wie der Name schon verrät, handelt es sich hierbei um eine Kombination aus öffentlichen und privater Cloud. Diese können kombiniert werden, um sicherheitsrelevante von allgemein zugänglichen Services zu trennen.
\end{enumerate}

Die im Cloud Computing angebotenen Eigenschaften sind dabei sehr hilfreich für die Robotik. Vor allem spielt die Rechenintensive Auswertung der Daten eine wichtige Rolle. Roboter können dabei gesammelte Informationen direkt an Servern in der Cloud schicken die diese auf verschieden Arten auswerten können oder für die weitere Verarbeitung zwischenspeichern können. Weitere Anwendungen und dahingehende Details werden im Laufe dieser Arbeit behandelt.
