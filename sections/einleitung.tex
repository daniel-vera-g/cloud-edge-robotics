\section{Einleitung}
\label{sec:Einleitung}

Die Robotik und die Themenbereiche rund um die Cloud und das Edge Computing sind sehr relevant für die Zukunft.\\
Die Robotik bietet dabei eine große Bandbreite an Anwendungsmöglichkeiten. Diese erstrecken sich von regulären Haushaltsrobotern bis hin zum Einsatz in speziellen Industriellen Anwendungsgebieten. Dies zeigt sich eindrucksvoll an der erst im Oktober 2022 veröffentlichten Statistik der International Federation of Robotics in der die Nutzung von Robotern im Professionellen Kontext um  37\% im Jahre 2021 gestiegen ist \cite{ifrSalesRobotsService2022} . Im Haushaltskontext gab es dabei eine Erhöhung von 12\% in den Verkaufszahlen.\\
Die Cloud und speziell der Bereich des Edge Computing spielen im Alltag und im Kommerziellen Umfeld ebenfalls eine entscheidende Rolle. Die Cloud bietet dabei die Möglichkeit auf unterschiedliche Computer Ressourcen über das Internet zuzugreifen und diese bei Bedarf einfach zu skalieren \cite{mellNISTDefinitionCloud} . Das Edge Computing fokussiert sich dabei auf die Anwendungsfälle am Rande des Netzwerks und ist ein wichtiger Bestandteil in der Umsetzung von Themenbereichen wie das Autonome Fahren oder Smart Cities.

Neben denen im letzten Absatz angesprochenen Themenbereiche, profitiert auch die Robotik von der Entwicklung im Bereich des Cloud und Edge Computing. Im speziellen, kann es hier die Ressourcen in der Cloud oder Edge nutzen, um die Funktionalität zu verbessern oder die gesammelten Daten besser zu nutzen.\\
Die in dieser Arbeit thematisierten Aspekte, orientieren sich an dem von Milan Groshev et al. (2015) \cite{groshevEdgeRoboticsAre2022} evaluierten Zustandes im Cloud und Edge Robotics Bereich. Im Folgenden, wird die Motivation hinter der Kombination beider Technologien erläutert, die Herausforderungen definiert und eine Übersicht über die untersuchten Lösungsansätze präsentiert die man in der Praxis nutzen kann.

\subsection{Struktur der Arbeit} % (fold)
\label{sub:Struktur der Arbeit}

Diese Arbeit ist in drei Teilen aufgebaut. Im ersten Teil wird eine Hinführung auf die für diese Arbeit nötigen Grundlagen gegeben. Dazu werden Konzepte aus den Bereichen der Cloud und dem Edge eingeführt. Es werden ebenfalls die Eigenschaften von Robotik Systemen beschrieben und mit dem \acrlong{ros} das aktuell gängige System zur Nutzung in der Robotik vorgestellt. Um die Schnittstelle zum Cloud und Edge Bereich herzustellen, wird hier ein Fokus auf die verwendeten Kommunikationsprotokolle in der Robotik gesetzt.\\
Der Hauptteil Gliedert sich in vier Abschnitte. Zu Beginn wird erklärt, wie die Zusammenführung von der Cloud, dem Edge und Robotern funktionieren kann und welche Vorteile man aus dem Zusammenschluss bekommen kann. Im folgenden, wird ein Blick auf die Einbindung der Roboter in das Gesamtsystem durch das Zenoh Protokoll gesetzt. Nach den eher spezielleren Themen, wird in den letzten beiden Abschnitten vom Hauptteil eine übergeordnete Sicht auf das Thema Cloud und Edge Robotics gemacht. Dazu werden allgemeine Architekturen betrachtet. Darauf aufbauend, werden verschiedene Use Cases vorgestelt die eine praktische Anwendung zeigen sollen.\\
Die Arbeit wird mit dem Fazit abgeschlossen. Hier wird ein Ausblick auf Technologien und die Weiterentwicklung in der Forschung und Entwicklung von Bereichen im Cloud und Edge Robotics Spektrum gemacht. Schließlich wird das ganze durch ein Fazit über die behandelten Themen beendet.

% subsection Struktur der Arbeit (end)



\subsection{Motivation}

Wie man der Einleitung entnehmen konnte, steigt die Nutzung von Robotern kontinuierlich. Vor allem in der Industrie stellt sich dabei die Frage, wie man eine große Anzahl an Robotern in Produktion skalieren, in instabilen Umgebungen produktiv betreiben oder mit etablierten Cloud Lösungen integrieren kann. Beim Versuch diese Fragen zu beantworten, stößt man zwangsläufig auf Herausforderungen die im Zusammenhang mit der Nutzung von Robotern untereinander oder in Verbindung mit externen Services stehen.\\
Bei den genannten Herausforderungen geht es um verschiedene Bereiche. Zum einen natürlich um die zusammenarbeit Heterogener Systeme untereinander. Diese müssen fähig sein, miteinander interagieren zu können und zu kommunizieren. Wenn man noch die Cloud und Edge Server miteinbezieht, kommen noch weitere Systeme dazu die man berücksichtigen muss. Zum anderen, kommen hier auch die Cloud und Edge Systeme durch ihre unterschiedlichen Eigenschaften zur Geltung. Im Cloud Bereich, hat man dabei homogene Ressourcen die kurze Zyklen durchlaufen. Außerdem hat man eine hohe Verfügbarkeit. Im Edge Bereich ist genau das Gegenteil der Fall. Ressourcen sind sehr Heterogen, haben lange Lebenszyklen und die Verfügbarkeit ist sehr Variabel.\\
Neben der Art der Ressourcen, sind auch die Eigenschaften des Netzwerkes eine konstante Herausforderung. Ziel ist es dabei einen hohen Durchsatz sowie eine niedrige Latenz zu haben. Dies ist aber aufgrund der Art des Netzwerkes oft nicht möglich. Vor allem im Bereich der Robotik, muss man mit Drahtlosen, unzuverlässige Verbindungen umgehen.\\
Zu den Netzwerkherausforderungen kommen schließlich noch die Herausforderungen die sich durch die Topologie ergeben. Diese ergeben sich durch die große Anzahl an Beweglichen Knoten die im System vorhanden sind und die Geographische Verteilung die diese haben können.\\
Die Fragestellung dieser Arbeit setzt sich also aus den hier geschilderten Fragen zusammen. Dabei ist zu klären wie man mit den verschiedenen Herausforderungen umgehen kann und mit welcher Lösung man die verschiedenen Komponenten am besten in Einklang bringt.\\

